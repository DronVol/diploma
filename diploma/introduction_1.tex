\section{Введение}

\par Одной из важных современных отраслей механики сплошной среды является теория фильтрации. Прикладное значение знаний, полученных во время изучения этой дисциплины, обусловлено большим значением энергетических и природных ресурсов, необходимых для жизни современного человечества. Такие отрасли, как нефтедобыча и очистка воды от примесей, являются сегодня одними из лидирующих в мире.
\par В большинстве природных процессов фильтрация жидкости сквозь пористую среду происходит при наличии в жидкости мелких в масштабах течения частиц. Примерами таких течений могут быть течение загрязнённой воды сквозь очищающий слой фильтра, закачка пропанта в трещину после её создания методом гидроразрыва пласта.
\par Характерными особенностями таких задач являются движение мелких частиц в относительно высоких концентрациях, которое происходит со скоростями, равными скорости несущего флюида, и загрязнение фильтрующего пласта, которое сопровождается изменением пористости и других параметров.
\par Целью работы является изучение таких важных процессов для описания такой задачи, как отложение частиц на поверхности пористого скелета и их диффузия в процессе движения. Также исследуются различные модели течения жидкости и оседания частиц на поверхности порового пространства.
\par  В данной работе была рассмотрена одномерная модель загрязнения пласта, где загрязнение моделируется кинетическим уравнением, отложение частиц зависит от скорости жидкости, пористости и концентрации частиц, а также были получены решения для функций пористости и концентрации частиц.