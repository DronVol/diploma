\section{Решение задачи без упрощений}
\subsection{Полное уравнение.}
\par Теперь получим решение в случае, когда уравнение неразрывности не упрощается. Имеем систему 
\begin{equation*}
\begin{cases}
(m\alpha)_{t}+u_{0}\alpha_{x}&= m_{t}\\
m_{t}&= -\gamma u_{0}\alpha\\
\end{cases}
\end{equation*}
\par Подставляя второе в первое и раскрывая скобки, получим $$\d \alpha_{t}+\frac{u_{0}}{m}\alpha_{x}=(\alpha-1)\gamma\frac{u_{0}}{m}\alpha$$
\par Рассматривая характеристики $\d \frac{dx}{dt}=\frac{u_{0}}{m}$ и интегрируя соотношение, получаем $\d \int \frac{d\alpha}{\alpha^{2}-\alpha}=\int\gamma dx=\int\frac{d(\alpha-1)}{\alpha-1}-\int\frac{d\alpha}{\alpha}$. Интегрируя и потенцируя, получаем окончательно выражение для $\alpha$:
\begin{equation*}
\d
\alpha=
\begin{cases}
\d
\frac{1}{1-\frac{\alpha_{0}-1}{\alpha_{0}}e^{\gamma x}},\;\;& x<\frac{u_{0}}{m}t\\
0,\;\;& x\geq \frac{u_{0}}{m}t\\
\end{cases}
\end{equation*}

\par Теперь, зная решение для $\alpha$, можно выразить $m$: $ \d
m= m|_{t=0}+\int\limits^{t}_{0}m_{t}dt=m_{0}+\int\limits^{t}_{0}\gamma u_{0}\alpha=m_{0}+\int\limits^{t}_{\frac{xm_{0}}{u_{0}}}\gamma u_{0}\frac{1}{1-\frac{\alpha_{0}-1}{\alpha_{0}}e^{\gamma x}} dt=$
\begin{equation*}
=
\begin{cases}
\d
m_{0}-\gamma u_{0}\frac{1}{1-\frac{\alpha_{0}-1}{\alpha_{0}}e^{\gamma x}}(t-\frac{xm_{0}}{u_{0}}),\;\;& x<\frac{u_{0}}{m_{0}}t\\
m_{0},\;\;& x\geq \frac{u_{0}}{m_{0}}t\\
\end{cases}
\end{equation*}

\par Можно заметить, что полученное решение при $\alpha_{0} \rightarrow 0$ переходит в полученное ранее выражение $$\d \alpha=\frac{1}{1-\frac{\alpha_{0}-1}{\alpha_{0}}e^{\gamma x}}\approx\frac{\alpha_{0}}{e^{\gamma x}}$$