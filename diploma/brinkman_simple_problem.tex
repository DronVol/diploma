\section{Задача о течении в высокопористой среде}
\par Рассмотрим подробнее задачу о течении в высокопористой среде, в которой используется уравнение Бринкмана.
\par Рассмотрим течение вязкой несжимаемой жидкости сквозь высокопористую среду. Пусть частицы могут осаждаться на стенки пористой среды. Тогда получим следующую систему уравнений:
\begin{equation*}
\begin{cases}
div\;\vec{u}= 0\\
-\vec{\nabla}p-\frac{\mu}{k}\vec{u}+\mu_{1}\Delta\vec{u}=0\\
(m\alpha)_{t}+div(\alpha\vec{u})=m_{t}+\cancelto{0}{D\Delta\alpha}\\
m_{t}=-\gamma \alpha |\vec{u}|\\
\end{cases}
\end{equation*}
\par В этой задаче полагаем, что засорение скелета мало, все коэффициенты --- постоянные.
\par Здесь мы рассмотрим следующую ситуацию: пусть скорость направлена по оси $x$, то есть
$$\vec{u}=u_{0}(y)\vec{e}_{x}$$
\par В этом приближении можем записать уравнение 
$$m_{0}\alpha'_{t}+u_{0}\alpha'_{x}=m_{t}+D(\cancelto{\text{\begin{scriptsize}
учёт диффузии в длинном тонком канале
\end{scriptsize}}}{\alpha''_{xx}}+\alpha''_{yy})$$
\par При $D=0$ получаем известное уже нам решение $\alpha = \alpha_{0}exp(-\gamma x)$, которое не зависит от $y$. Также, мы можем исследовать уравнение границы загрязнения:
$$\frac{dx_{f}}{dt}=\frac{u_{0}(y)}{m_{0}}$$
$$x_{f}=\frac{u_{0}(y)}{m_{0}}t=x_{f}(y,t)$$
\par Из последнего соотношения виден замечательный факт: со временем поверхность всё больше "размазывается".
\par Проверим условия на скачке: 
\begin{equation*}
\begin{cases}
[m]=0\\
[m(\alpha-1)]D-[\alpha u_{n}]=0\\
[u_{n}]=0\\
\end{cases}
\end{equation*}
\par Найдём скорость распространения границы $f = x-\frac{u_{0}(y)}{m_{0}}t=0$:
$$D=-\frac{f_{t}}{|\vec{\nabla}f|}=\frac{-\frac{u_{0}(y)}{m_{0}}}{\sqrt{1+(\frac{u'_{0}(y)}{m_{0}}t)^{2}}}=\frac{u_{0}}{\sqrt{m_{0}^{2}+(u'_{0}(y)t)^{2}}}$$
\par Найдём выражение для вектора нормали к поверхности раздела с "чистой" фазой:
$$\vec{n}=\frac{(1,-\frac{u_{0}'(y)}{m_{0}}t)}{\sqrt{1+(\frac{u'_{0}(y)}{m_{0}}t)^{2}}}=\frac{(m_{0}, -u'_{0}(y)t)}{\sqrt{m_{0}^{2}+(u'_{0}(y)t)^{2}}}$$
\par Используя полученные соотношения, проверим второе условие на скачке:
$$m_{0}[\alpha]\frac{u_{0}}{\sqrt{m_{0}^{2}+(u'_{0}(y)t)^{2}}}-[\alpha]\frac{u_{0}(y)m_{0}}{\sqrt{m_{0}^{2}+(u'_{0}(y)t)^{2}}}=0$$
\par Отсюда видно, что условие баланса массы автоматически выполняется.
\par Теперь получим само решение для $u_{0}(y)$. Запишем систему уравнений в декартовой системе координат, обозначив $\vec{u}=(u_{x}, u_{y})$:
\begin{equation*}
\begin{cases}
\d \frac{\partial u_{x}}{\partial x}+\cancelto{0}{\frac{\partial u_{y}}{\partial y}}=0\\
\d -\frac{\partial p}{\partial x}-\frac{\mu}{k}u_{x}+\mu_{1}\cancelto{0}{\frac{\partial^{2} u_{x}}{\partial x^{2}}}+\mu_{1}\frac{\partial^{2} u_{x}}{\partial y^{2}}=0\\
\d -\frac{\partial p}{\partial y}-\frac{\mu}{k}\cancelto{0}{u_{y}}+\mu_{1}\cancelto{0}{\frac{\partial^{2} u_{y}}{\partial x^{2}}}+\mu_{1}\cancelto{0}{\frac{\partial^{2} u_{y}}{\partial y^{2}}}=0\\
\end{cases}
\end{equation*}
\par Отсюда видно, что:\\
\par 1) Скорость зависит только от $y$.\\
\par 2) Давление зависит только от $x$.\\
\par Во втором уравнении перенесём давление в одну часть, а скорости в другую, то есть:
$$\frac{\partial p}{\partial x}=-\frac{\mu}{k}u_{x}+\mu_{1}\frac{\partial^{2} u_{x}}{\partial y^{2}}$$
\par Тогда видно, что левая и правая часть зависят от разных переменных. Это значит, что обе они одновременно равны одной и той же постоянной, которую мы обозначим $A$. Отсюда получаем, что давление линейно зависит от $x$:
$$P=Ax+C$$
\par Выпишем уравнение для профиля скорости:
$$\frac{\partial^{2} u_{x}}{\partial y^{2}}-\frac{\mu}{k\mu_{1}}u_{x}-\frac{A}{\mu_{1}}=0$$
\par Решение этого уравнения выглядит следующим образом:
$$u_{x}(y)=-\frac{Ak}{\mu}+C_{1}exp\;(\sqrt{\frac{\mu}{\mu_{1}k}y})+C_{2}exp\;(-\sqrt{\frac{\mu}{\mu_{1}k}y})$$
\par Величины $C_{1}$ и $C_{2}$ находятся из граничных условий. В этой задаче их два:\\
\par 1) Условие симметрии на расстоянии $h$ (ищем выражение для течения в канале)
$$\frac{\partial u_{x}}{\partial y}|_{y=h}=0$$
\par 2) Условие Навье проскальзывания на границе $y=0$: 
$$\frac{\partial u_{x}}{\partial y}|_{y=0}=bu_{x}|_{y=0}$$
\par Отсюда находим $C_{1}$ и $C_{2}$. Обозначим $\alpha=\frac{\mu}{\mu_{1}k}$. Тогда:
$$C_{1}=\frac{Abk}{h(b-\alpha+e^{2h\alpha}(b+\alpha))}$$
$$C_{2}=\frac{Abk}{h(b+\alpha+e^{-2h\alpha}(b-\alpha))}$$
\par [Построить графики с характерными параметрами, (b~0.1h)]
