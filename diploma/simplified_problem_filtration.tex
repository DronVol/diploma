\section{Решение упрощённой задачи с конкретным уравнением засорения}
\subsection{Дополнительные предположения}
\par Предположим, что $m\approx m_{0}$, то есть изменение пористости можно считать малым. Также можно предположить, что $\alpha\ll 1$ --- концентрация мала. Тогда от уравнения $$m_{t}\alpha + m\alpha+u_{0}\alpha_{x}=m_{t}$$ можно перейти к $$m_{0}\alpha_{t}+u_{0}\alpha_{x}=m_{t}$$ Добавим в систему частный случай кинетического уравнения засорения: $m_{t}=-\gamma \alpha |u_{0}|$, где множитель $\alpha |u_{0}|$ пропорционален объёму жидкости, протекающей через данную точку, а $\gamma=const$ --- экспериментальный коэффициент, который, например, в общем случае может зависеть от пористости. Далее мы проверим, что система имеет решение $\alpha=\alpha(x)$ за фронтом (не зависит от $t$, однако $m$ от времени зависит).\\
\par В итоге получаем уравнение $$\alpha'_{t}+\frac{u_{0}}{m_{0}}\alpha'_{x}=-\frac{\gamma\alpha |u_{0}|}{m_{0}}$$ 
\par Уравнению засорения соответствует характеристика $\d \frac{dx}{dt}=const$, а уравнению выше --- $\d\frac{dx}{dt}=\frac{u_{0}}{m_{0}}$. Поскольку разрыв идёт вдоль второй характеристики, то от разрыва уходит только одна характеристика --- условие эволюционности выполняется. Выпишем соотношение на разрыве: $[m(1-\alpha)]D-[(1-\alpha)u_{n}]=0,$ где $u_{n}=u_{0}$. Тогда: $m_{0}[1-\alpha]\frac{u_{0}}{m_{0}}-[1-\alpha]u_{0}=0$. Здесь мы пользуемся гипотезой $[m]=0$, но возможны и другие ситуации. \\
\par Можно поставить следующую задачу: при заданной $\alpha|_{x=0}=\alpha_{0}$ найти решение $\alpha(x)$ за скачком, а также $m=m(x,t)$. 
\subsection{Решение поставленной задачи}
\par Найдём решение уравнения вдоль характеристики $\d \frac{dx}{dt}=0$: $$\d \alpha_{t}+\frac{u_{0}}{m_{0}}\alpha_{x}=-\frac{\gamma\alpha|u_{0}|}{m_{0}}$$ 
\par Отсюда получаем $\d \frac{d\alpha}{dt}=-\frac{\gamma\alpha|u_{0}|}{m_{0}}$. $\d \int\frac{d\alpha}{\alpha}=-\int\frac{\gamma|u_{0}|}{m_{0}}dt$, откуда окончательно получаем $$\d ln\;\alpha=-\frac{\gamma|u_{0}|}{m_{0}}t+C$$\\
\par Перепишем полученное решение в виде $\d ln\frac{\alpha}{\alpha_{0}}=-\frac{\gamma|u_{0}|}{m_{0}}(t-t_{0})$. Так же, вспоминая, что мы искали решение вдоль характеристики, из $\d \frac{dx}{dt}=\frac{u_{0}}{m_{0}}$ можем получить $\d x=\frac{u_{0}}{m_{0}}(t-t_{0})$. Это показывает, что найденное решение можно представить в виде $\alpha=\alpha_{0}e^{-\gamma x}$, или 
\begin{equation*}
\d
\alpha=
\begin{cases}
\d
\alpha_{0}e^{-\gamma x},\;\;& x<\frac{u_{0}}{m_{0}}t\\
0,\;\;& x\geq \frac{u_{0}}{m_{0}}t\\
\end{cases}
\end{equation*}
 Представляя это решение в виде $\d \alpha=\alpha_{0}e^{-\gamma \frac{u_{0}}{m_{0}}(t-t_{0})}$, можно получить, что скорость разрыва $\d D=\frac{u_{0}}{m_{0}}$. Проверим выпонение условия баланса массы на скачке: $[m(1-\alpha)]D-[(1-\alpha)u_{n}]=0$, откуда $m_{0}[1-\alpha]D-[1-\alpha]u_{n}=0$. Так как $m_{0}D-u_{n}=0$, то отсюда следует $u=u_{0}=const$\\
\par Величина $\gamma$ имеет размерность $[\gamma]=\textbf{1/м}$. Её можно трактовать как типичную длину засорения, так как величина $\alpha/\alpha_{0}$ падает в $e$ раз на расстоянии $1/\gamma$ от начала координат.\\
\par Также у нас осталось уравнение $m_{t}=-\gamma u_{0}\alpha$. С его помощью найдём решение 
\begin{equation*}
\d
m=m|_{t=0}+\int\limits^{t}_{0}m_{t}dt=m_{0}-\int\limits^{t}_{0}\gamma u_{0}\alpha dt=
\begin{cases}
\d
m_{0}-\gamma u_{0}\alpha(x)(t-\frac{xm_{0}}{u_{0}}),\;\;& x<\frac{u_{0}}{m_{0}}t\\
m_{0},\;\;& x\geq \frac{u_{0}}{m_{0}}t\\
\end{cases}
\end{equation*}