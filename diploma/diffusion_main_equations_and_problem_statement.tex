\section{Диффузия частиц в потоке в отсутствие оседания}
\subsection{Основные понятия}
\par Далее рассмотрим другой эффект: диффузию частиц. Как показано в \cite{phillips}, это может происходить по нескольким причинам.
\par Пусть $N_{b}$ --- поток частиц, связанный с броуновским движением, который пропорционален градиенту концентрации.
\par $N_{\mu}$ --- поток частиц, связанный с разницей в вязкости в различных слоях жидкости.
\par $N_{c}$ --- поток частиц, связанный со столкновением с другими объектами. В некоторых задачах рассматривается столкновение с частицами, мы же будем рассматривать столкновение с пористым скелетом.
\par Вообще говоря, механизмы диффузии крайне разнообразны. Их модели частично опираются на выводы из элементарной теории, частично  получаются эмпирически. В данной работе используется эмпирический подход, то есть строится не противоречащая наблюдаемым в повседневной жизни явлениям модель, которая далее будет применяться для получения простых аналитических результатов.
\par Далее будут рассмотрены задачи в средах, похожих на вату или сетку, в которых пористость велика. Такое приближение рассматривается ввиду того, что диффузия в низкопористых средах является моделью, полученной в результате осреднения более сложных процессов, механизм которых сильно отличается от диффузии в течении жидкости. Будем считать, что геометрия такова, что не может препятствовать миграции частиц в каком-либо направлении (среда изотропна).
\pagebreak
\section{Основные формулы и уравнения фильтрации с диффузией}
\subsection{Уравнение движения Бринкмана}
\par Ввиду того, что в рассмотрение вводятся среды, в которых пористость полагается большой, следует рассмотреть уравнение движение из \cite{brinkman}, которое будем называть законом фильтрации (или уравнением) Бринкмана: $$0 = - \vec{\nabla}p-\frac{\mu}{k}\vec{u}+\mu_{1}\Delta\vec{u}$$
\par Основная причина рассмотрения другого уравнения движения заключается в том, что закон Дарси рассматривает только действие пористого скелета на жидкий объём $(\sim \mu v/k)$, но не учитывает трение между слоями жидкости. По этой причине в уравнении движения оставляется член, связанный с вязким трением, а коэффициент $\mu_{1}$ определяется в \cite{brinkman} по следующей формуле: $$\mu_{1}=\mu(1+2{,}5 (1-m))$$.
\subsection{Компоненты общего потока диффузии}
\par Как уже говорилось выше, в этой работе рассматривается модель диффузии, которая может быть условно разделена на три составляющие:
\par 1) Броуновское движение частиц, связанное с множеством факторов, в том числое с хаотичностью устройства вутренней геометрии среды, которая была до получения осреднённых величин в теории фильтриции. $$\vec{N}_{b} = -D\vec{\nabla} \alpha$$
\par 2) Диффузия, связанная со столкновением частиц с пористым скелетом. Получена из следующего соображения: этот член диффузии линейно зависит от вариации числа столкновений частиц со скелетом, поторая в свою очередь пропорцианальна градиенту скорости частиц. Таким образом можем записать, приняв за $d$ характерный размер частиц пористого скелета или волокон:
$$\vec{N}_{c} = -Kd\vec{\nabla}(|u|\alpha)$$
\par 3) Диффузия, связанная с переменной вязкостью (!!!!!!!!!!!!!!!!!!!!!!!!!!!!!!!!!!!!)

