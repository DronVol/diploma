\section{Решения в виде рядов}
%\subsection{Различия с решённой задачей}
\par Существует метод поиска точных решений путём разложения общего уравнения в ряд по концентрации осаждённых частиц. Если полагать, что скорость отложения частиц не просто пропорциональна  модулю скорости фильтрации, а есть некоторая функция пористости, то система переписывается следующим образом.

\par Обозначим концентрацию осаждённых частиц $s = (1-m_{0})-(1-m)$. Тогда нашу систему можно записать как:

\begin{equation*}
\begin{cases}
(a(s)\alpha+s)_{t}+(b(s)\alpha)_{x}&= 0\\
-(s)_{t}&=K(s)\alpha\\
\end{cases}
\end{equation*}

\par Сопоставим коэффициенты, зависящие от $s$ с обозначениями в решённых задачах. $a(s)$ --- $m$, $b(s)$ --- $u_{0}$, $K(s)$ --- $-\gamma u_{0}$. Полученные этим способом функции $s$ и $\alpha$ обобщают решение, полученное в рассмотренных предельных случаях. 

\par Полагая $K(s)=\varepsilon\Lambda(s)$, где $\varepsilon$ --- малый параметр и раскаладывая функции $a(s)$, $b(s)$ и $\Lambda(s)$ в ряды по $s$ в окрестности нуля. Это справедливо, так как всё ещё принята гипотеза о слабом засорении. Таким образом получим системы дифференциальных уравнений на коэффициенты рядов, решая которые можно получить решения необходимой точности в данном классе решений.
\par Решения ищутся в виде $s(x,t,\varepsilon)=\varepsilon s_{1}(x,t)+\varepsilon^{2} s_{2}(x,t)+...$
\par Выпишем эти системы для нескольких членов разложения:

\begin{equation*}
\varepsilon^{1}: 
\begin{cases}
(a_{0}\alpha_{1}+a_{1}\alpha_{0}s_{1})_{t}+(b_{0}\alpha_{1}+b_{1}\alpha_{0} s_{1})_{x}&= -\lambda_{0} \alpha_{0}\\
-(s_{1})_{t}&= - \lambda_{0} \alpha_{0}\\
\end{cases}
\end{equation*}

\begin{equation*}
\varepsilon^{2}: 
\begin{cases}
(a_{0}\alpha_{2}+a_{1}\alpha_{1}s_{1}+a_{2}\alpha_{0} s_{1}^{2}+a_{1}\alpha_{0}s_{2})_{t}+\\+(b_{0}\alpha_{2}+b_{1}\alpha_{1} s_{1}+b_{2}\alpha_{0}s_{1}^{2}+b_{1}\alpha_{0} s_{2})_{x}&= -(\lambda_{0} \alpha_{1}+\lambda_{1}\alpha_{0}s_{1})\\
-(s_{2})_{t}&= - (\lambda_{0} \alpha_{1}+\lambda_{1}\alpha_{0}s_{1})\\
\end{cases}
\end{equation*}

\par Решая последовательно эти системы методом характеристик можно получить искомое решение с необходимой наперёд заданной точностью.