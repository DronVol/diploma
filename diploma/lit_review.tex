\section{Обзор литературы}
\subsection{Фильтрация с осаждением}
\par В природе процессы фильтрации \cite{barenblatt} и \cite{basniev} чаще всего протекают при наличии в текущем флюиде примесей, размеры которых много меньше, чем размеры пор породы. Участие этих примесей в засорении может сказываться как на течении жидкости как таковой, влияя на её эффективные механические параметры, так и на объёмной доле пор, что также меняет картину течения. 
\par В статьях \cite{osiptsov_1} и \cite{osiptsov_2} рассмотрено влияние сопротивления жидкости и силы Архимеда на распределение частиц в ламинарном потоке несущей жидкости в течении внутри трещены гидроразрыва в приближении тонкого слоя. Распределение частиц полагается неоднородным, рассматривается двухфазная модель течения. В ходе работы проводится сравнение численной модели с упрощённой аналитической и с экспериментаьными данными.
\par В статье \cite{kuzmina} рассматриваются различные долгосрочные эффекты влияния оседания примесей на пористый скелет, а также предлагается метод анализа подобных задач на больших временах с использованием рядов. В качестве результата получается асимптотическое решение задачи вблизи фронта концентрации.
\par В статье \cite{civan} рассматриваются различные виды зависимости скорости изменения пористого скелета из-за  осаждения частиц на его поверхности и выноса их оттуда потоком флюида. Также в статье предлагается обобщение данных моделей при помощи эмпирических зависимостей. Уравнения основываются на уравнениях сплошной среды и статистических закономерностях, обобщающих эмпирические законы.
\subsection{Фильтрация с диффузией}
\par В статье \cite{brinkman} исследуется течение в высокопористых средах и обосновывается закон движения в подобном типе задач. Производится сравнение с экспериментальными данными и более простыми зависимостями.
\par В статье \cite{leighton} рассматривается влияние диффузии сферических частиц на эффективную вязкость суспензии как целого. Выводится аналитическая модель и сравнивается с экспериментальными данными.
\par В статье \cite{phillips} приводится описание исследования различных эффектов, связанных с диффузией частиц в потоке высококонцентрированной смеси, на различные характеристики течения. Рассматриваются две различные постановки задачи: течение Куэтта между двумя соосными вращающимися цилиндрами и течение в трубе под действием перепада давления. В этой работе выделяется три основных компоненты, описывающие диффузионный поток: хаотическое движение частиц, соударение между частицами и диффузия из более вязких слоёв в менее вязкие. Полученные зависимости сравниваются с экспериментальными данными.
